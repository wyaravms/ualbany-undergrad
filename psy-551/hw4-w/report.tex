\documentclass[a4paper, 12pt]{article}

%

% Margin - 1 inch on all sides
%
\usepackage[margin=1in]{geometry}


%
% Babel package for multiple languages typesetting
%
\usepackage[english]{babel}

% Doublespacing
%
\usepackage{setspace}
\doublespacing

%
% Setting the font

%\usepackage{times}

%
% Rotating tables (e.g. sideways when too long)
%
\usepackage{rotating}


%
% For multiples rows in tables
% 
\usepackage{multirow}

\usepackage{graphicx}
\DeclareGraphicsExtensions{.pdf,.png,.jpg}
\graphicspath{ {./img/} }

\usepackage{caption}
\usepackage{subcaption}

\usepackage{amsmath}
%\usepackage{mathtools}


\makeatletter% Set distance from top of page to first float
\setlength{\@fptop}{5pt}
\makeatother

\setlength\abovedisplayskip{0pt}

% 
%Line numbering in verse environment
%
%\usepackage{lineno}
\usepackage{enumerate}
%
%Fancy-header package to modify header/page numbering (insert last name)
%
%\usepackage{fancyhdr}
%\pagestyle{fancy}
%\lhead{}
%\chead{}
%\rhead{\thepage}
%\lfoot{}
%\cfoot{}
%\rfoot{}
%\renewcommand{\headrulewidth}{0pt}
%\renewcommand{\footrulewidth}{0pt}
%To make sure we actually have header 0.5in away from top edge
%12pt is one-sixth of an inch. Subtract this from 0.5in to get headsep value
\setlength\headsep{0.333in}


\begin{document}
\begin{flushleft}

%%%%First page name, class, etc
Wyara Vanesa Moura e Silva\\
Professor Kevin Knuth \\
PHY/CSI/INF 451/551: Bayesian Data Analysis / Signal Processing \\
Oct 16, 2014\\

%\noindent
%\hrulefill
%%%%Title
\begin{center}
\textbf{Homework 4 Written}
\end{center}
%\noindent
%\hrulefill 


\setlength{\parindent}{0.5in}

\textbf{1. Bayes Theorem and the Weather.}

\textbf{(a) Hypothesis and the data } 

\begin{table}[ht]
    \centering
    \begin{tabular}{l l}
        Hypothesis: & $R$ - On days when it does rain  \\
                    & $\overline{R}$ - On days when it does not rain\\
										& \\
        Data:       & $DC$ - There are dark clouds that roll in during the morning \\
                    & $\overline{DC}$ - There are not dark clouds that roll in during the morning \\
    \end{tabular}
    \label{tab:hypdata}
\end{table}


\textbf{(b) Bayes Theorem for this problem} 

$$ p(R | DC, I) = \frac{p(DC | R, I)p(R | I)}{p(DC|I)} $$

\textbf{(c) Expression for the evidence}

$$ p(DC | I) = p(DC, R | I) + p(DC, \overline{R} | I) = $$
%$$ p(RO | I)p(DC | RO,I) + p(\neg RO | I)p(DC | \neg RO, I) $$


\textbf{(d) Using Bayes Theorem to solve the problem} 

\begin{table}[ht]
    \centering
    \begin{tabular}{l l}
        $ p(R | I) = 0.29 $    & $ p(\overline{R} | I) = 0.71 $ \\
        $ p(DC | R, I) = 0.9 $ &  $ p(DC | \overline{R}, I) = 0.25 $ \\
    \end{tabular}
\end{table}

$$ p(RO | DC, I) = \frac{p(DC | R, I)p(R | I)}{p(DC|I)} $$
$$ = \frac{p(DC | R, I)p(R | I)}{p(DC | R,I)p(R | I) + p(DC | \overline{R}, I)p(\overline{R} | I)} $$
$$ = \frac{0.9 \times 0.29}{0.9 \times 0.29 + 0.25 \times0.71 } $$
$$ = \frac{0.261}{0.4385} = 0.59521  $$
%$$ p(RO | DC, I) = \frac{0.29 \times 0.9}{0.4385} = 0.59521 $$ 



\textbf{2. You are a laptop repair person.}

\textbf{(a) Hypothesis and the data } 

\begin{table}[ht]
    \centering
    \begin{tabular}{l l}
        Hypothesis: & $FP$ - Laptop’s power supply has failed  \\
                    & $\overline{FP}$ - Laptop’s power supply is OK\\
										& \\
        Data:       & $S$ - Plugging it in will produce smoke \\
                    & $\overline{S}$ - Plugging it doesn't in will produce smoke \\
    \end{tabular}
    \label{tab:hypdata}
\end{table}


\textbf{(b) Bayes Theorem for this problem} 

$$ p(FP | S, I) = \frac{p(S | FP, I)p(FP | I)}{p(S|I)} $$

\textbf{(c) Expression for the evidence}

$$ p(S | I) = p(S, \overline{FP} | I) + p(S, FP | I) = $$
%$$ p(S | FP,I)p(FP | I) + p(S | \overline{FP}, I)p(\overline{FP} | I) $$


\textbf{(d) Using Bayes Theorem to solve the problem} 

\begin{table}[ht]
    \centering
    \begin{tabular}{l l}
        $ p(FP | I) = 0.3 $    & $ p(\overline{FP} | I) = 0.7 $ \\
        $ p(S | FP, I) = 0.45 $ &  $ p(S | \overline{FP}, I) = 0.05 $ \\
    \end{tabular}
\end{table}

$$ p(FP | S, I) = \frac{p(S | FP, I)p(FP | I)}{p(S|I)} $$
$$ = \frac{p(S | FP, I)p(FP | I)}{p(S | FP,I)p(FP | I) + p(S | \overline{FP}, I)p(\overline{FP} | I)} $$
$$ = \frac{0.45 \times 0.3}{0.45 \times 0.3 + 0.05 \times0.7 } $$
$$ = \frac{0.135}{0.17} = 0.7941	  $$
%$$ p(RO | DC, I) = \frac{0.29 \times 0.9}{0.4385} = 0.59521 $$ 


\textbf{3. The blue M\&M was introduced in 1995. }

\begin{table}[ht]
    \centering
    \begin{tabular}{l l}
        Symbols: & $Y$ - Yellow M\&Ms  \\
                    & $G$ - Green M\&Ms\\
										& \\
										& $B1=94$ - Bag 1 from 1994 \\
                    & $B2=96$ - Bag 2 from 1996 \\
										& $B1=94$ - Bag 1 from 1994 \\
										& $B2=96$ - Bag 2 from 1996 \\
    \end{tabular}
    \label{tab:hypdata}
\end{table}


\begin{table}[ht]
    \centering
    \begin{tabular}{l l}
        $ p(B1=94, B2=96 | I) = 0.5 $    & $ p(B1=96, B2=94 | I) = 0.5  $ \\
    \end{tabular}
\end{table}

\textbf{Solution using Bayes Theorem}

$$ p(YB1, GB2 | B1=94, B2=96, I ) = 0.2 \times 0.2 = 0.04 $$
$$ p(YB1, GB2 | B1=96, B2=94, I ) = 0.14 \times 0.1 = 0.014 $$

\textbf{Bayes Expression}

If B1 is from 1994, so B2 is from 1996. Therefore, I will not put that B2=94 or B2=96 in the probabilities condition for not confuse.


$$ p(B1=94 | Y B1, G B2, I) = \frac{p(YB1, G B2 | B1=94, I)p(B1=94 | I)}{p(Y B1, G B2|I)} $$
$$ = \frac{p(YB1, G B2 | B1=94, I)p(B1=94 | I)}{p(YB1, GB2 | B1=94,I)p(B1=94 | I) + p(YB1, G B2 | B1=96, I)p(B1=96| I)} $$
$$ = \frac{0.04 \times 0.5}{0.04 \times 0.5 + 0.014 \times0.5 } $$
$$ = \frac{0.02}{0.027} = 0.7407	  $$



\end{flushleft}

\end{document}