\documentclass[a4paper, 12pt]{article}

%

% Margin - 1 inch on all sides
%
\usepackage[margin=1in]{geometry}


%
% Babel package for multiple languages typesetting
%
\usepackage[english]{babel}

% Doublespacing
%
\usepackage{setspace}
\doublespacing

%
% Setting the font

%\usepackage{times}

%
% Rotating tables (e.g. sideways when too long)
%
\usepackage{rotating}


%
% For multiples rows in tables
% 
\usepackage{multirow}

\usepackage{graphicx}
\DeclareGraphicsExtensions{.pdf,.png,.jpg}
\graphicspath{ {./img/} }

\usepackage{caption}
\usepackage{subcaption}

\usepackage{amsmath}
%\usepackage{mathtools}


\makeatletter% Set distance from top of page to first float
\setlength{\@fptop}{5pt}
\makeatother

\setlength\abovedisplayskip{0pt}

% 
%Line numbering in verse environment
%
%\usepackage{lineno}
\usepackage{enumerate}
%
%Fancy-header package to modify header/page numbering (insert last name)
%
%\usepackage{fancyhdr}
%\pagestyle{fancy}
%\lhead{}
%\chead{}
%\rhead{\thepage}
%\lfoot{}
%\cfoot{}
%\rfoot{}
%\renewcommand{\headrulewidth}{0pt}
%\renewcommand{\footrulewidth}{0pt}
%To make sure we actually have header 0.5in away from top edge
%12pt is one-sixth of an inch. Subtract this from 0.5in to get headsep value
\setlength\headsep{0.333in}


\begin{document}
\begin{flushleft}

%%%%First page name, class, etc
Wyara Vanesa Moura e Silva\\
Professor Kevin Knuth \\
PHY/CSI/INF 451/551: Bayesian Data Analysis / Signal Processing \\
Sept 30, 2014\\

%\noindent
%\hrulefill
%%%%Title
\begin{center}
\textbf{Homework 2 Written}
\end{center}
%\noindent
%\hrulefill 


\setlength{\parindent}{0.5in}

\textbf{1. A fair 20-sided die}

\textbf{(a)} States: \{1\}, \{2\}, \{3\}, \{4\}, \{5\}, \{6\}.

All of them are mutually exclusive because two or more events cannot happen at the same time, and exhaustive because at least one of those outcomes will occur.

\textbf{(b)} As the fair 20-sided die, so all probability of which side is the same (p). Therefore, the sum of probabilities of all state is 1.

$$ \displaystyle\sum_{i=1}^{20}p(i=x|I)=1 \rightarrow \displaystyle\sum_{i=1}^{20}p=1 \rightarrow 20p=1 \rightarrow p=\frac{1}{20} $$

So, the probability of $i=7$ is equal: 

$$ p(i=7|I)=\frac{1}{20} = 0.05 $$

\textbf{c)} If $i$ is odd $\rightarrow$ there are 10 sided odd.

$$ p(i \; is \; odd|I) = p(i=1 \vee i=3 \vee i=5 \vee i=7 \vee i=9 \vee i=11 \vee $$
$$ i=13 \vee i=15 \vee i=17 \vee i=19|I) $$
$$ p(i \; is \; odd|I) = p(i=1|I) + p(i=3|I) + p(i=5|I) + p(i=7|I) + p(i=9|I) +  $$
$$ p(i=11|I) + p(i=13|I) + p(i=15|I) + p(i=17|I) + p(i=19|I) $$
$$ p(i \; is \; odd|I) = 10p = 10 \times \frac{1}{20} = \frac{1}{2} = 0.5 $$

\textbf{d)} If $i$ is prime $\rightarrow$ there are 8 sided prime.

$$ p(i \; is \; prime|I) = p(i=2 \vee i=3 \vee i=5 \vee i=7 \vee $$
$$ i=11 \vee i=13 \vee i=17 \vee i=19|I) $$
$$ p(i \; is \; prime|I) = p(i=2|I) + p(i=3|I) + p(i=5|I) + p(i=7|I) + $$
$$ p(i=11|I) + p(i=13|I) + p(i=17|I) + p(i=19|I) $$
$$ p(i \; is \; prime|I) = 8p = 8 \times \frac{1}{20} = \frac{2}{5} = 0.4 $$

\textbf{e)} The expected value of $i$ is calculated for:

$$ E(i) = \displaystyle\sum_{i=1}^{20} i \times P(x=i|I) $$
$$ E(i) = 1 \times \frac{1}{20} + 2 \times \frac{1}{20} + 3 \times \frac{1}{20} + 4 \times \frac{1}{20} + 5 \times \frac{1}{20} + 6 \times \frac{1}{20} + 7 \times \frac{1}{20} + $$
$$ 8 \times \frac{1}{20} + 9 \times \frac{1}{20} + 10 \times \frac{1}{20} + 11 \times \frac{1}{20} + 12 \times \frac{1}{20} + 13 \times \frac{1}{20} + 14 \times \frac{1}{20} + $$
$$  15 \times \frac{1}{20} + 16 \times \frac{1}{20} + 17 \times \frac{1}{20} + 18 \times \frac{1}{20} + 19 \times \frac{1}{20} + 20 \times \frac{1}{20} $$
$$ E(i) = 10.5 $$

\textbf{f)} No, it is not possible to observe the expected value because it is not an integer, and can be regarded as weighted average.

\textbf{g)} 
$$ p(i \; is \; \{1,2,3,4,5\}|I) = p(i=1 \vee i=2 \vee i=3 \vee i=4 \vee i=5) $$
$$ p(i=1|I) + p(i=2|I) + p(i=3|I) + p(i=4|I) + p(i=5|I) $$
$$ 5p = 5 \times \frac{1}{20} = \frac{1}{4} $$
$$ p(i \; is \; \{1,2,3,4,5\}|I) = 0.25 $$


\textbf{2. Independent Pair of Fair 8-Sided Dice}

\textbf{a)} Considering the independence of the dice, the probability of the state $j$ occur does not dependent of the state $i$ occur.

$$ p(i|j,I) = \frac{p(i|j,I)}{p(j|I)} = \frac{p(i|I)p(j|I)}{p(j|I)} = p(i|I)$$

\textbf{b)} The two dice are independent, so the first dice have 8 sides with 8 possible events (\{1\}, \{2\}, \{3\}, \{4\}, \{5\}, \{6\}, \{7\}, \{8\}).

$$ \displaystyle\sum_{i=1}^{8} =1 \rightarrow \displaystyle\sum_{i=1}^{8}p=1 \rightarrow 8p=1 \rightarrow p=\frac{1}{8} $$

Therefore,
$$ p(i=2|I) = p = \frac{1}{8}$$

\textbf{c)} 
$$ p(i=2, j=4|I) = p(i=2|I) = p(j=4|i=2|I) $$
$$ \frac{1}{8} \times p(j=4|I) = \frac{1}{8} \times \frac{1}{8} = \frac{1}{64} $$

\textbf{d)} The expected value of $i$:
$$ E(i) = \displaystyle\sum_{i=1}^{8} i \times P(x=i|I) $$
$$ E(i) = 1 \times \frac{1}{8} + 2 \times \frac{1}{8} + 3 \times \frac{1}{8} + 4 \times \frac{1}{8} + 5 \times \frac{1}{8} + 6 \times \frac{1}{8}+ 7 \times \frac{1}{8} + 8 \times \frac{1}{8}$$
$$ E(i) = \frac{36}{8} = 4.5 $$

\textbf{e)} No, it is not possible to observe the expected value because it is not an integer, and can be regarded as weighted average.

\textbf{f)} Expected value of $i+j$:
$$ E(i+j) = \displaystyle\sum_{i=1}^{8}\displaystyle\sum_{i=1}^{8} (i+j) \times P(i,j|I) $$
$$ E(i+j) = \displaystyle\sum_{i=1}^{8}\displaystyle\sum_{i=1}^{8} (i+j) \times P(i|I) \ast p(j|I) $$
$$ E(i+j) = \displaystyle\sum_{i=1}^{8}\displaystyle\sum_{i=1}^{8} (i+j) \times \frac{1}{8} \ast \frac{1}{8} $$
$$ E(i+j) = \displaystyle\sum_{i=1}^{8}\displaystyle\sum_{i=1}^{8} (i+j) \times \frac{1}{64} $$


All possible sum are in the Table 1 below: 

\begin{table}[ht]
    \centering
    \begin{tabular}{| c | c | c | c | c | c | c | c | c | }
        \hline
        i / j    & 1 & 2  & 3  & 4  & 5  & 6 & 7 & 8 \\
        \hline
        1   & 2 & 3  & 4  & 5  & 6  & 7 & 8 & 9\\
        \hline
        2   & 3 & 4 & 5 & 6 & 7  & 8 & 9 & 10\\
        \hline
        3   & 4 & 5 & 6 & 7  & 8 & 9 & 10 & 11\\
        \hline
        4   & 5 & 6 & 7  & 8 & 9 & 10 & 11 & 12\\
        \hline
        5   & 6 & 7  & 8 & 9 & 10 & 11 & 12 & 13\\
        \hline
        6   & 7 & 8  & 9  & 10  & 11 &12 & 13 & 14\\
        \hline
				7   & 8 & 9  & 10  & 11  & 12 &13 & 14 & 15\\
        \hline
				8   & 9 & 10  & 11  & 12  & 13 &14 & 15 & 16\\
        \hline
    \end{tabular}
    \caption{Table with all the possible cases of sum}
    \label{tab:allprob}
\end{table}
$$ E(i) = 1 \times \frac{2}{64} + 2 \times \frac{3}{64} + 3 \times \frac{4}{64} + 4 \times \frac{5}{64} + 5 \times \frac{6}{64} + 6 \times \frac{7}{64} + $$
$$ 7 \times \frac{8}{64} + 8 \times \frac{9}{64} + 7 \times \frac{10}{64} + 6 \times \frac{11}{64} + 5 \times \frac{12}{64} + 4 \times \frac{13}{64} +  $$
$$ 3 \times \frac{14}{64} + 2 \times \frac{15}{64}+ \frac{16}{64}$$
$$ E(i) = \frac{2}{64} + \frac{6}{64} + \frac{12}{64} + \frac{20}{64} + \frac{30}{64} + \frac{42}{64} + \frac{56}{64} + \frac{72}{64} + \frac{70}{64} + \frac{66}{64} + \frac{60}{64} $$
$$ + \frac{52}{64} + \frac{42}{64} + \frac{30}{64} + \frac{16}{64} $$
$$ E(i) = \frac{576}{64} = 9$$ 

\textbf{g)} There are 8 possibilities to happen sum equal 9.
$$ p(i+j=9|I) = \frac{8}{64} = 0.125 $$

\textbf{3. Coupled Dice}

\textbf{a)} Probabilities of all the possible cases.
$$ p(i=1|I) = 0 \ \textrm{and} \ p(j=6|I) = 0 $$

$$ P(i=x, j=x|I) = \underbrace{2p(i=x,j=y|I)}_{adjacent \ \textrm{faces}} = \underbrace{4p(i=x,j=y|I)}_{opposite \ \textrm{faces}}  $$
$$ p = 2p_{a} = 4p_{o} $$

The same faces appear:
$$ \displaystyle\sum_{i=2}^{5} p(i=x, j=x|I) $$
$$ p(i=2, j=2|I) + p(i=3, j=3|I) + p(i=4, j=4|I) + p(i=5, j=5|I) $$
$$ p + p + p + p = 4p $$

Adjacent faces appear: 
$$ p(i=x, j=x|I) = 2p(i=x, j=y|I) $$
$$ 4p = 2p_{a} $$
$$ 2p = p_{a} $$

Opposite faces appear:
$$ p(i=x, j=x|I) = 4p(i=x, j=y|I) $$
$$ 4p = 4p_{a} $$
$$ p = p_{a} $$

\textbf{b)} Proof of how the probabilities sum to unity:

$$ \displaystyle\sum_{i=1}^{6}\displaystyle\sum_{i=1}^{6} p(i=x, j=y|I) = 1 $$
$$ 4 \times 4p + 8 \times 2p + 4 \times p = 1 $$
$$ 36p = 1 \rightarrow p = \frac{1}{36} $$


The same faces appear: $4p = \frac{1}{9}$ 

Adjacent faces appear: $2p = \frac{1}{18}$

Opposite faces appear: $p = \frac{1}{36}$

All probabilities can be viewed in the Table 2 below:

\begin{table}[ht]
    \centering
    \begin{tabular}{| c | c | c | c | c | c | c | }
        \hline
        i / j    & 1 & 2  & 3  & 4  & 5  & 6 \\
        \hline
        1   & 0 & 0  & 0  & 0  & 0  & 0 \\
        \hline
        2   & 0 & 4p & 2p & 2p & p  & 0 \\
        \hline
        3   & 0 & 2p & 4p & p  & 2p & 0 \\
        \hline
        4   & 0 & 2p & p  & 4p & 2p & 0 \\
        \hline
        5   & 0 & p  & 2p & 2p & 4p & 0 \\
        \hline
        6   & 0 & 0  & 0  & 0  & 0  & 0 \\
        \hline
    \end{tabular}
    \caption{Table with probabilities for all the possible results}
    \label{tab:allprob}
\end{table}

\textbf{c)} The expected value of i+j:
$$ E(i+j) = \displaystyle\sum_{i=1}^{6}\displaystyle\sum_{i=1}^{6} (i+j) \times P(i,j|I) $$

All possible sums are below in the Table 3:
\begin{table}[ht]
    \centering
    \begin{tabular}{| c | c | c | c | c | c | c | }
        \hline
        i / j    & 1 & 2  & 3  & 4  & 5  & 6 \\
        \hline
        0   & 0 & 0 & 0  & 0  & 0  & 0 \\
        \hline
        0   & 3 & 4 & 5 & 6 & 7  & 0 \\
        \hline
        0   & 4 & 5 & 6 & 7  & 8 & 0 \\
        \hline
        0   & 5 & 6 & 7  & 8 & 9 & 0 \\
        \hline
        0   & 6 & 7  & 8 & 9 & 10 & 0 \\
        \hline
        0   & 0 & 0  & 0  & 0  & 0 &0 \\
        \hline
    \end{tabular}
    \caption{Table with all the possible cases of sum}
    \label{tab:allprob}
\end{table}
$$ E(i+j) = \underbrace{4 \times 4p + 6 \times 4p + 10 \times 4p +}_{\textrm{the same faces appear}} $$
$$ \underbrace{5 \times 2p + 5 \times 2p + 6 \times 2p + 8 \times 2p + 8 \times 2p + 9 \times 2p + 9 \times 2p +}_{\textrm{adjacent faces appear}} $$
$$ \underbrace{7 \times p + 7 \times p + 7 \times p + 7 \times p}_{\textrm{opposite faces appear}} $$

$$ E(i+j) = 28 \times 4p + 56 \times 2p + 28 \times p = 112p + 112p + 28p $$
$$ E(i+j) = 252p = 252 \times \frac{1}{36} $$
$$ E(i+j) = 7 $$

\textbf{d)} $p(i|I)$ for all values of $i$.

$$ \displaystyle\sum_{i=1}^{6}p(i|I) = p(1|I) + p(2|I) + p(3|I) + p(4|I) + p(5|I) + p(6|I) $$
$$ p(2|I) \rightarrow \textrm{for which row is the same} $$
$$ p(2|I) = 4p + 2p + 2p + p = 9p = 9 \times \frac{1}{36} = \frac{1}{4}$$

\textbf{e)} Expected value of $i$.
$$ E(i) = \displaystyle\sum_{i=1}^{6}i \times P(i|I) = 1 \times 0 + 2 \times \frac{1}{4} + 3 \times \frac{1}{4} + 4 \times \frac{1}{4} + 5 \times \frac{1}{4} + 6 \times 0   $$
$$ E(i) = \frac{1}{2} + \frac{3}{4} + 1 + \frac{5}{4} = \frac{2+3+4+5}{4} = \frac{14}{4} $$ 
$$ E(i) = 3.5 $$



\end{flushleft}

\end{document}